\chapter{Introduction}
\label{chap:intro}

\section{Networks and systems pervade}
Shed some light on the role networks play in our lives, highlighting different types of networks.

\section{Electricity costs in networks and systems} 
The significance of network electricity costs amongst various sources of operational costs in networks of different types~\cite{brill:DataCenterCrisis:UI:2007}. Let the numbers speak for themselves.

\section{Energy inefficiency characterizes today's networks} Networks are not energy proportional and provisioned for peak workload, therefore, they are not energy efficient. This means that operators are spending way more on electricity costs than they ideally should. 

\section{Prevalent energy efficiency improvement techniques} \textit{Electricity cost} = \textit{amount of energy consumed} $\times$ \textit{unit price of electricity}. Therefore, electricity cost may be cut by reducing either or both of the quantities on the right hand side. 

\subsection{Reducing the amount of energy consumed}
\begin{enumerate}
\item Deploy hardware with better energy efficiency. But that is a capital-intensive and possibly disruptive option. Our focus is on improving electricity cost for operational legacy networks.
\item The role of virtualization in reducing the energy consumption in data centers.
\item Amount of energy consumed may also be reduced using Resource Pruning (RP)
\end{enumerate}

\subsection{Using cheaper electricity}
Using resources at locations with cheaper electricity, i.e., Workload Relocation (WR).

\section{Our thesis} Our thesis is that RP and WR may be combined into a unified optimization framework that can reduce electricity cost for different types of networks.

\section{Contributions} This thesis makes the following contributions:

\section{Organization} The rest of the document is structured as follows