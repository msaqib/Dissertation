\chapter{Background - Different Types of Networks and Their Similarities }
\label{chap:background}
In this chapter we look at two different network types and observe operational similarities between them from the point-of-view of power consumption.
\section{Geo-Diverse Data Centers} Data centers host applications that we consume every day. Operators such as Amazon, Google and Microsoft deploy data centers that are geographically dispersed for (i) fault tolerance (ii) low-latency to the clients.
\subsection{Structure}
A really basic introduction covering: composition of a typical data center (racks, pods, networking, cooling etc)
\subsection{Request routing} front-end server based load balancing and request routing mechanisms such as IP Anycast
\subsection{Power consumption model} Describe the power consumption model from prior work and derive a more simplified yet equivalent model

\section{Cellular Networks} Just as data centers enable applications that we rely on every day, cellular networks are an important enabler of another pervasive service: telephony. 
\subsection{Structure} A really basic introduction to cellular networks covering: concept of cells, mobile stations (MSs), Base Transceiver Stations (BTSs) and Base Station Controllers (BSCs) 
\subsection{Call placement} How a call is handled by a BTS (at a very abstract level, i.e., how is the serving BTS chosen). Role of the BSC in cell association and call hand-off
\subsection{Power consumption model} Describe the power consumption model from prior work

\section{Similarities between different types of networks} Geo-diverse data centers and cellular networks are similar in the sense that both are built out of network resources to handle workload which results in electricity consumption. 