\chapter{Conclusions and Future Work}
\label{chap:conclusions}
\section{Contributions} In this thesis, we have made the following contributions:
\begin{itemize}
\item We model networks as a set of resources that have user workload handling capacity. When a resource handles workload, some power is consumed. The power consumption is an affine function of workload. On this basis, we identified similarities and contrasts between two different types of networks: geo-diverse data centers and cellular networks.
\item We model the electricity cost minimization problem in networks as a multi-interval optimal state trajectory problem.
\item We provide a mathematical optimization formulation for the optimal state trajectory problem. The formulation is parametrized and abstract for broad applicability. We modeled state transition costs as being a fraction of the cost of a state in a single interval.
\item We apply the mathematical optimization problem to geo-diverse data centers as well as cellular networks with the following contrasts:
	\begin{itemize}
	\item In cellular networks, the number of workload units being handled simultaneously is much smaller compared to geo-diverse data centers. For this reason, finding the optimal fractional distribution of workload to a set of resources is practically acceptable for geo-diverse data centers but not to cellular networks. This is because a fractional mapping of workload to resources may represent a call being handled by multiple BTSs simultaneously. Meanwhile in the case of geo-diverse data centers, a reasonably small fraction of the typical cumulative hourly workload (millions of client requests or more) is still a whole number so a fractional workload distribution does not imply a single client request being split over multiple data centers.
	\item A call may only be handled by a restricted set of nearby BTSs. In contrast, in a geo-diverse data center setting, it is common for applications to be replicated across data centers and in such cases, a client request may be handled at any data center.
	\item Geo-diverse data centers are so far apart that geographic diversity in electricity prices is quite apparent. Meanwhile, BTSs in cellular networks are not too distant and geographic diversity in electricity prices is not present.
	\item The transition costs in geo-diverse data centers are expected to be significant. However, in cellular networks, the electricity cost impact of resource activation and deactivation is negligible.
	\end{itemize}
\item We show that the electricity cost minimization problem is NP-Complete in both geo-diverse data center and cellular network scenarios and provide heuristic algorithms for solution of the problem in each of these networks. We were also able to solve reasonably sized problems for both network types.
\item We studied the sensitivity of the state trajectory problem to variations in the parameter values such as the magnitude of transition costs relative to the cost of a state in a single interval.
\end{itemize}

\section{Limitations} Like all research work, our work has some limitations. To the best of our knowledge, the following list best covers these:
\begin{itemize}
\item We only considered resource activation and deactivation costs as the source of transition costs. Data replication costs in geo-diverse data centers have not been evaluated due to (to the best of our knowledge) lack of models in the literature. 
\item We have not experimented with real-time electricity prices for geo-diverse data centers.
\item We have not considered costs associated with maintaining replication across all data centers.
\item RED-BL computes the hourly traffic volume to be mapped to each data center. It does not provide a mapping of this workload to individual servers within the data center. We have not focused on this because this is a complementary and specialized area of research which is presently receiving significant research attention.
\item We have not considered the ability of GSM BTSs to handle calls at the half-rate codec, which effectively increases a BTSs call handling capacity.
\end{itemize}

\section{Future work} Some of the avenues for future inquiry related to our work are:

\begin{itemize}
\item Study the inter-data center traffic to examine if there is a relationship between the volume of such traffic with the number/nature of client requests, duration of the interval for which data volume is measured or some other variables. This may help build a model for inter-data center traffic, which is quite expensive. Such a model may be integrated with RED-BL to have a more elaborate optimization framework that is sensitive to the potential increase in inter-data center traffic due to data center elastic resource (de)activation.
\item Replication of data stores across data centers requires some overhead traffic. An empirical study could be performed to build a model for such traffic in a few representative scenarios such as news websites, social networking sites, micro-blogging etc. Our work saves electricity cost by turning off elastic resources. This may mean taking the application data stores offline. When the elastic resources come back online, they would need to bring their data stores in sync with the rest of the data centers. The results of the aforementioned empirical study could be used to predict the volume of traffic that would be generated during this re-synchronization event.
\item The deployment of a small-scale GSM testbed using open source GSM software could be done to validate the results of our simulation study.
\item RED-BL may be applied to other networks such as generation resource scheduling in smart grids or packet switching networks.
\end{itemize}
