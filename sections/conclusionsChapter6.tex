\chapter{Conclusions and future work}
\label{chap:conclusions} Geo-diverse data centers and cellular networks may be abstractly represented as a set of resources. Each of these resources is constrained by the maximum amount of workload it can handle. Furthermore, the instantaneous power consumption for these resources is an affine function of the amount of workload they handle. The no-load power consumption for the resources in these networks is a large fraction of the full-load power consumption. 

The workload is quite variable, and the network must be dimensioned according to peak workload demand. However, the workload peaks for only a short duration and drops to a much lower trough. Thus, these networks are energy inefficient and their electricity cost is quite high. One way to improve the energy efficiency and save electricity costs is to scale the networks resources in response to workload variations.

We represent the status (on or off) of network resources and the workload mapped to them as a network state. Using this representation, we model the energy efficiency improvement problem as an optimal state trajectory problem, which we call Relocate Energy Demand to Better Locations (RED-BL). We used workload traces collected from real networks and real electricity costs to assess the utility of RED-BL. In doing so, we have made the following contributions:
\begin{itemize}
\item We identified abstractions for a generic power consumption model applicable to two different networks: geo-diverse data centers and cellular networks. Our model represents these networks as a set of resources that have an associated workload handling capacity. When a resource handles workload, some power is consumed. The power consumption for both of these networks is an affine function of workload. 
\item The aggregate mapping of workload to resources may be viewed as a network state. We accordingly model the electricity cost minimization problem in networks as a multi-interval optimal state trajectory problem.
\item We provide a mathematical optimization formulation for the optimal state trajectory problem. The formulation is parametrized and abstract for broad applicability. We modeled state transition costs as being a fraction of the cost of a state in a single interval.
\item We apply the mathematical optimization problem to geo-diverse data centers as well as cellular networks with the following contrasts:
	\begin{itemize}
	\item The optimal mapping of workload to resources is a discrete optimization problem. The workload capacity of a single resource is quite large in case of geo-diverse data centers, hence any fraction of workload mapped to a resource is likely to be quite close to a whole number of client requests. If this is not the case, the number of client requests mapped to a resource may be rounded to the nearest integer. The difference in power consumption resulting from this rounding is expected to be quite small. Hence, the optimal mapping of workload to resources, which is a discrete optimization problem may be relaxed to a fractional problem without much error in case of geo-diverse data centers. In cellular networks, on the other hand, the number of workload units being handled simultaneously by a single resource is much smaller. Thus, a fractional mapping of workload to resources in a cellular network may represent a call being handled by multiple BTSs simultaneously, which does not make sense. Hence, a fractional relaxation to the optimization problem is not possible in case of cellular networks.
	\item A call may only be handled by a restricted set of nearby BTSs. In contrast, in a geo-diverse data center setting, it is common for applications to be replicated across data centers and in such cases, a client request may be handled at any data center.
	\item Geo-diverse data centers are so far apart that geographic diversity in electricity prices is quite apparent. Meanwhile, BTSs in cellular networks are not too distant and geographic diversity in electricity prices is not present.
	\item The transition costs in geo-diverse data centers are expected to be significant. However, in cellular networks, the electricity cost impact of resource activation and deactivation is negligible.
	\end{itemize}
\item We show that the electricity cost minimization problem is NP-Complete in both geo-diverse data center and cellular network scenarios and provide heuristic algorithms for solution of the problem in each of these networks. We were also able to solve reasonably sized problems for both network types.
\item We studied the sensitivity of the state trajectory problem to variations in the parameter values such as the magnitude of transition costs relative to the cost of a state in a single interval.
\end{itemize}

\section{Scope of our work} Like all research work, our work's scope is not unlimited. To the best of our knowledge, the following list covers the scope of our work:
\begin{itemize}
\item We only considered resource activation and deactivation costs as the source of transition costs. Data replication costs in geo-diverse data centers have not been evaluated because of (to the best of our knowledge) lack of models in the literature. Coming up with a general model for this cost is beyond the scope of this thesis.
\item For geo-diverse data centers, we have experimented with day ahead electricity prices for the US market. 
\item RED-BL computes the hourly traffic volume to be mapped to each data center. It does not provide a mapping of this workload to individual servers within the data center. This is a complementary research problem, which is presently receiving significant research attention.
\item In case of cellular networks, the ability to handle greater traffic volume through the half-rate codecs has not been considered.
\end{itemize}

\section{Future work} Some of the avenues for future inquiry related to our work are:

\begin{itemize}
\item Study the inter-data center traffic to examine if there is a relationship between the volume of such traffic with the number/nature of client requests, duration of the interval for which data volume is measured, or some other variables. This may help build a model for the expensive inter-data center traffic. Such a model may be integrated with RED-BL to have a more elaborate optimization framework that is sensitive to the potential increase in inter-data center traffic due to data center elastic resource (de)activation.
\item Replication of data stores across data centers requires some overhead traffic. An empirical study could be performed to build a model for such traffic in a few representative scenarios such as news websites, social networking sites, micro-blogging etc. Our work saves electricity cost by turning off elastic resources. This may mean taking the application data stores offline. When the elastic resources come back online, they would need to bring their data stores in sync with the rest of the data centers. The results of the aforementioned empirical study could be used to predict the volume of traffic that would be generated during this re-synchronization event.
\item The deployment of a small-scale GSM testbed using open source GSM software could be done to validate the results of our simulation study.
\item RED-BL may be applied to other networks such as generation resource scheduling in smart grids or packet switching networks.
\end{itemize}
