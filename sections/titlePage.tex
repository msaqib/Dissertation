%%%%%%%%%%%%%%%%%%%%%%%%%%%%%%%%%%%%%%%%%%%%%%%%%%%%%%%%%%%%%%%%%%%%%%%%%%%
%%%%%%%%%%%%%%%%           The title page           %%%%%%%%%%%%%%%%%%%%%%%
%%%%%%%%%%%%%%%%%%%%%%%%%%%%%%%%%%%%%%%%%%%%%%%%%%%%%%%%%%%%%%%%%%%%%%%%%%%
\newpage
\thispagestyle{empty}
\begin{center}
  \vspace*{0.2in}
  {\LARGE \bf Leveraging Workload Relocation and Resource Pruning for Electricity Cost Minimization in Service Provider Networks}\\
  \vspace*{0.5in}
  {\Large \bf PhD Thesis}\\

  \vspace*{0.4in}
  {\Large\bf Muhammad Saqib Ilyas}\\
    \vspace*{0.2in}
  {\Large 2005-06-0024}\\
  \vspace*{0.4in}
  {\Large\bf Advisor: Zartash Afzal Uzmi}\\

	\vspace*{0.4in}
  \begin{center}
   \includegraphics[scale = 0.5]{./pics/lums-logo.eps}
  \end{center}
  \vspace*{0.4in}  
  {\Large\bf Department of Computer Science} \\
  \vspace*{0.2in}
  {\Large\bf Syed Babar Ali School of Science and Engineering} \\
  \vspace*{0.2in}
  {\Large \bf Lahore University of Management Sciences}
  
  
\end{center}
%%%%%%%%%%%%%%%%%%%%%%%%%%%%%%%%%%%%%%%%%%%%%%%%%%%%%%%%%%%%%%%%%%%%%%%%%%%
%%%%%%%%%%%%%%%% The dedication page, of you have one  %%%%%%%%%%%%%%%%%%%%
%%%%%%%%%%%%%%%%%%%%%%%%%%%%%%%%%%%%%%%%%%%%%%%%%%%%%%%%%%%%%%%%%%%%%%%%%%%
\newpage
\thispagestyle{empty}
\begin{center}
 \vspace*{3in}
  \textit{\LARGE {Dedicated to dedication}}\\

\end{center}

%%%%%%%%%%%%%%%%%%%%%%%%%%%%%%%%%%%%%%%%%%%%%%%%%%%%%%%%%%%%%%%%%%%%%%%%%%%
%%%%%%%%    Signature Page   Here                                 %%%%%%%%%
%%%%%%%%%%%%%%%%%%%%%%%%%%%%%%%%%%%%%%%%%%%%%%%%%%%%%%%%%%%%%%%%%%%%%%%%%%%
\newpage
\thispagestyle{empty}
\begin{center}
  \vspace*{0.8cm}
\textbf{\Large Lahore University of Management Sciences}\\
\vspace*{0.8cm} \textbf{\large School of Science and
Engineering}\\\vspace*{0.8cm} \textbf{\large CERTIFICATE}
\end{center}
\vspace*{0.5cm}I hereby recommend that the thesis prepared under my
supervision by \textbf{\textit{Muhammad Saqib Ilyas}} titled
\textbf{\textit{Leveraging Workload Relocation and Resource Pruning for Electricity Cost Minimization in Service Provider Networks}} be accepted in partial fulfillment of the requirements for the degree of doctor of philosophy in computer engineering.\vspace{0.3in}
\begin{flushright}
Zartash Afzal Uzmi (Advisor) \end{flushright}
\textbf{\underline{Recommendation of Examiners' Committee:}}\\
\\\textbf{Name} \hspace*{5.5cm} \textbf{Signature}\\ \\
Zartash Afzal Uzmi \hspace*{2.7cm} {---------------------}\\\\
Ihsan Ayyub Qazi \hspace*{3.0cm} {---------------------}\\\\
%Y \hspace*{1.7cm} {---------------------}\\\\
Tariq Mehmood Jadoon\hspace*{2.1cm} {---------------------}\\\\
Muhammad Fareed Zaffar\hspace*{1.8cm} {---------------------}

%%%%%%%%%%%%%%%%%%%%%%%%%%%%%%%%%%%%%%%%%%%%%%%%%%%%%%%%%%%%%%%%%%%%%%%%%%%
%%%%%%%%    Acknowledgements Here                                 %%%%%%%%%
%%%%%%%%%%%%%%%%%%%%%%%%%%%%%%%%%%%%%%%%%%%%%%%%%%%%%%%%%%%%%%%%%%%%%%%%%%%
\newpage
\thispagestyle{empty}
\begin{center}
  \vspace*{1cm}
  \textbf{\large Acknowledgements}
\end{center}

\addcontentsline{toc}{chapter}{\numberline{}Acknowledgements}
Back in 2003, I realized that I wanted to get a PhD. With the help of a friend and colleague, Syed Abbas Ali, I looked around myself in Karachi for PhDs in computer networks. I found few. I went to seek advice from Dr. Irfan Hyder of PAF KIET, Karachi. That was a very useful visit. He gave me two books. One was the Harvard manual of dissertation writing and the other was titled ``How to get a PhD". From the latter, I discovered certain things about research and PhD that came in handy later. Apparently, reading that book didn't help much, as it still took me more than a decade to defend my PhD thesis (pun on myself intended).


I got into the PhD program at LUMS in 2005 and selected Dr. Zartash as my PhD advisor. Later on, I met a few people who had attempted to get him as an advisor but failed. His response to them was an impossible task, such as reading an entire book and seeing him the next evening. They never went back. He asked me no such thing. I recall that he did ask me about the seriousness of purpose, but nothing else. I could never figure out what he saw in me that he did not challenge me at that point. Throughout my PhD, I saw his confidence in me only grow and I can't thank him enough for that. I never could find the reason for that confidence, but it certainly gave me an immense hope of eventual success. A few other excellent graduate and undergraduate students did work with him over the years, but I was the first to clear the PhD qualifying exam and continue on to defend the thesis. I hope that he graduates many more PhD students. He deserves the credit that would bring. Thank you, sir, for being yourself. I still vividly remember how you hugged me to congratulated on the acceptance of our paper in INFOCOM mini-conference. It is one of my life's most cherished moments because it reminds me that I brought you joy and pride. 


What a ride it has been! Really, what a roller coaster! It took over ten stressful years, but it was worth it. When I look back now, I see that I have gained a lot. I took a lot of breadth and depth courses which were enriching. Of course, I also learned to do research. In the end, having achieved all that, does it bring me a feeling of pride? No, on the contrary, it is a humbling experience. It only made me realize how small and insignificant I am. It only helped me realize how little I know, no matter how much I might learn. To me, that is one of the most important results of getting a PhD.


I am not a people person, more of a loner. But you can't get a PhD without valuable contributions from a lot of people. My fellow PhD students at LUMS were an immense help. I spent a lot of time with them. We ate together, we had tea together, we shared lively discussions as well as depressing ones. We played darts in the research lab. We even played an indoor version of cricket that was played with a ball removed from a computer mouse. Don't even ask me what the bat was made from. Aadil Zia Khan mastered the art of spinning that ball and was great with the bat, too. We played outdoor cricket in the nets, too, and it was splendid. We invented a thing called ``Shutup Weekend", inspired by Startup Weekend, where on a Friday evening, we made pledges of what we will accomplish by the coming Sunday evening. It would be all action and no talk. We were surprised by how much we could achieve. Later on, as some of us graduated and others gave in to pressures of life, our group grew apart, but we still keep in touch. I am not much of a support to people, but they all gave a lot of strength to me through our interaction. Given the tough conditions that we have all been through, I consider us LUMS PhDs to be the cockroaches\footnote{Cockroaches are known to survive in extremely tough conditions.} of research.


My parents have always been very understanding and have shown confidence in my decisions. I applied to MS programs in the US without telling them. When I got admissions, they only expressed joy and supported me. When I went for my Masters degree to the US, my sister and her husband were supportive from day one. They are very encouraging and supportive to this day. I am the youngest of three brothers, so I got a lot of attention. My brothers took care of me through playground injuries to college admissions. I stayed with my elder brother when I was a day scholar at LUMS. Later on, I got married and moved out, against his will. I am thankful to my parents and siblings for their never ending love, prayers and support.


My wife has accepted her fate of being at home without me, literally from day one since I left her at her parents' place and returned to LUMS the day after our wedding. Thanks to the pressures of graduate studies, we never went out much. I am sorry to have missed my eldest daughter Fatima's childhood. She is a very patient child but my lack of attention shows in her eyes sometimes. Her younger twin sisters are a completely different story. While it was easy to dodge Fatima and drown myself in graduate studies, these two are inescapable. They may happily shove me out the door every morning, but when I return, they catch me and make sure that I spend enough time with them. A big thanks to my wife and daughters.


A key person from whom I learnt the ropes during my early days in research is Fahad Rafique Dogar. He is an outstanding researcher. He was also instrumental in shaping part of my PhD thesis as well as a great source of advice and feedback on my thesis write-up and presentation. 


Saqib Raza was instrumental in the early stages of my PhD thesis and helped me forge a key collaboration with Chen-Nee Chuah and her research group at UC, Davis. My sincere gratitude to all my collaborators at UC, Davis. 


I was fairly proud of my teaching skills until I met Dr. Ihsan Qazi and learnt what great teaching means. He is an excellent teacher and an outstanding researcher. After Dr. Zartash, he has been the single biggest source of advice on my PhD thesis. I was also fortunate to learn how to prepare a course when I co taught a couple of courses with him. He is gifted with a never ending positive attitude. He shows you positives in every situation and encourages you. 


I met Dr. Ali Khayam for a few minutes in 2012 in Dr. Ihsan Qazi's office. Dr. Ali stood up to greet me when Dr. Ihsan Qazi told him that my paper had been accepted in the INFOCOM Mini-Conference. Later, when I learnt that Dr. Ali had a paper in the full conference paper in INFOCOM, I discovered the meaning of humility.

 
Apart from Dr. Zartash, many faculty members at LUMS have been supportive and helpful to me in one way or another. In no particular order, I would like to thank Dr. Sohaib Khan, Dr. Tariq Jadoon, Dr. Fareed Zaffar, Dr. Shahab Baqai, Dr. Naveed Arshad and Dr. Mian Muhammad Awais. 


My colleagues at Namal College, Mianwali have been very supportive and encouraging. I especially wish to thank Dr. Malik Jahan Khan, Dr. Junaid Akhtar and Dr. Adnan Iqbal for their understanding, prayers and support.


In the end, I want to thank the most important contributor to my success, Allah Almighty, who enabled me to do what was beyond my abilities. Alhamdulillah for giving me the patience and strength to emerge through extremely stressful situations that abound on the way to a PhD. This is not an end, but the beginning of a journey. I have not reached a milestone, but received the visa to embark on a journey of discovery. I pray to Allah to enable me to undertake a spectacular journey of research.