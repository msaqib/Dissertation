\chapter{A generalized framework for electricity cost optimization}
\label{chap:framework} In Chapter~\ref{chap:intro}, we have seen that many different types of networks are plagued by high electricity costs. If the networks were energy efficient, this would be justifiable due to high workload. However, many types of networks today are characterized by lack of energy proportionality and significant daily variations in workload which result in energy inefficiency. Hence, high electricity costs in networks is a major concern and minimizing electricity costs is an important research problem.  

Since electricity costs depend on two things: i) the amount of electricity consumed, ii) electricity prices. We developed insights into the former by investigating network operation as it relates to power consumption. In doing so, we observed that there are several similarities and a few minor differences between different network types in terms of power consumption. In this chapter, we will leverage the similarities in different types of networks to formulate an optimization problem that minimizes electricity costs for different types of networks. We will also comments on how the subtle differences between different network types will be incorporated in this optimization problem.


\section{Problem Model} %Discuss how different network types can periodically use RP and WR to minimize electricity costs. Show that this problem is NP-Hard. Describe the concept of network state and motivate a state trajectory optimization problem. Describe transition costs and formulate a mathematical optimization problem.
In order to develop a generalized model for network electricity cost and to formulate an optimization problem for minimizing electricity cost, we use an illustrative example. We will then comment on the complexity of the problem before developing an optimization problem formulation.

\subsection{Illustrative Example}
\label{subsec:framework:example}
Let us illustrate network operation from the standpoint of electricity consumption and cost using an example shown in Figure~\ref{fig:mappingexample}. The example uses a test tube to represent a network resource and marbles to represent a unit workload. The network resource would be a data center in the context of geo-diverse data center operator, whereas it would be a transceiver in the case of a cellular operator. Similarly, the workload unit would be a client request in the data center context, whereas it would be a call in a cellular network setting. The operator's goal is to assign workload to network resources and, if needed, periodically update this mapping in response to variations in workload.

We consider the largest possible quantum of time for which the workload (and electricity price) remains fixed and term each such quantum as an \textit{interval}. We assume that workload for several consecutive intervals is known and term this sequence of intervals as a planning window.   The example demonstrates three different ways of mapping this workload to two network resources situated at different locations. For simplicity we assume in this example that the workload is geographically split such that half of it originates near each of the two resource. For this example, we consider temporal variatioin in workload as shown in Figure~\ref{fig:mappingexample} (a). Meanwhile, Figure~\ref{fig:mappingexample} (b) shows the geo-temporal variation in electricity prices for the two network resources.


One possible operational strategy is to map each workload unit to the nearest available resource as shown in Figure~\ref{fig:mappingexample} (c). In a sense, this is the default strategy in cellular networks, whereby a call is handled by the BTS from which the mobile station (MS) receives the strongest radio signal\footnote{Signal from the physically nearest BTS may be weakened considerably due to natural or man-made obstructions. In such cases, the nearest BTS may not be the one from which the strongest signal is received. Hence, we take "nearest" to mean the BTS from which the MS receives the strongest signal}. In geo-diverse data center settings, this sort of mapping is also often the default strategy because it minimizes the access latency for all clients\footnote{Network latency has been shown to have a strong correlation with the physical shortest path distance between two locations on the globe~\cite{dina:p2pdelay:infocom:2004}. So, the commonly understood physical measure of "shortest" applies in this case.}. 

The above workload-resource mapping strategy pays no attention to geo-diversity in electricity prices. We can exploit geo-diversity in electricity prices to reduce the electricity cost over the planning window by mapping more workload to resources at cheaper locations. To this end, we must change the way workload is mapped to resources as the electricity prices at various locations changes. We term such changes in workload-resource mapping as Workload Relocation (WR). Figure~\ref{fig:mappingexample} (d) shows a mapping strategy that uses WR to map as much workload as possible to resources at cheaper locations. In interval $t_1$, since the cumulative workload equals the total network capacity, both network resources will be operating at capacity. Accordingly, there is not opportunity to reduce electricity costs using either WR or RP. In interval $t_2$, on the other hand, as shown in Figure~\ref{fig:mappingexample} (d), we may use WR to move all workload to network resource B, which is situated at the location with the cheapest electricity price, thereby reducing electricity cost for that interval as compared to the default workload-mapping strategy shown in Figure~\ref{fig:mappingexample} (c). While doing this WR, care must be taken not to violate the workload capacity of a network resource. In interval $t_3$, again, as shown in Figure~\ref{fig:mappingexample} (d) WR may be used to shift workload to network resource A to reduce electricity costs compared to the default strategy shown in at the location with cheapest electricity price during that interval.

Due to lack of energy proportionality in networks, the power consumption of idle resources is a large fraction of their peak power consumption. Hence, consolidation of workload to cheaper locations offers a limited benefit in terms of reducing electricity cost compared to the default workload-mapping strategy. To avail considerable savings in electricity cost, one must use resource pruning (RP), i.e., deactivate idle resources. Notice that in the deafult workload-resource mapping strategy of Figure~\ref{fig:mappingexample} (c), there is no opportunity to deactivate idle resources in any of the three intervals. However, the purely-WR strategy of Figure~\ref{fig:mappingexample} (d) may be augmented with RP, as shown in Figure~\ref{fig:mappingexample} (e), to achieve maximal savings in electricity cost. The strategy in Figure~\ref{fig:mappingexample} (e) not only shifts workload to the cheapest possible resources, but also deactivates as many resources as possible. 

In claiming that Figure~\ref{fig:mappingexample} (e) shows the maximal savings in electricity cost, we have assumed that activation and deactivation of network resources is free of cost. However, such costs may exist in practice and in some network types may even be significant compared to the total electricity cost of network operation. In such cases, care must be taken when defining the optimal strategy for network operation. With this in mind, we draw parallels with similar problems in other domains with known results on optimal solutions and hence draw conclusions on the computational complexity of the optimal electricity cost network operation.

\begin{figure}
\centering
\includegraphics[height=0.7\textheight]{pics/mappingexample.eps}
\caption{An example of mapping variable workload to capacity-limited network resources with geo-temporal diversity in electricity prices. Three consecutive intervals $t_1$, $t_2$ and $t_3$ are considered. Workload and electricity prices may only change between two consecutive intervals. (a) Workload considered in this example. (b) Electricity prices for the locations at which the two network resources are situated. (c) A uniform mapping of workload to network resources does not exploit electricity price diversity. (d) Mapping workload to network resources in order of their current electricity price. Due to lack of energy proportionality, only slight savings in electricity cost are possible. (e) Deactivating idle resources alongwith the resource mapping strategy of (d) may result in significant electricity cost savings.}
\label{fig:mappingexample}
\end{figure}



\subsection{Problem complexity}
\label{subsec:framework:complexity}
During each interval, the network operation problem maps to the multiple knapsacks problem ~\cite{kellerer:knapsackproblems:2005,Chekuri99aptas}, whereby a subset of a given set of items, each with a certain weight, must be selected and placed into several weight-limited knapsacks such that the total profit from the selected items is maximized. Since the single-interval instancce of our problem is analogous to the multiple knapsacks problem, which is known to be NP-Hard~\cite{kellerer:knapsackproblems:2005,Chekuri99aptas}, each single-interval instance of our problem is NP-Hard as well. Hence, the multi-interval planning problem must also be NP-Hard. This applies to cellular networks, for instance, where every call may be associated to exactly one BTS. 

In certain type of networks, workload may be fractionally distributed, i.e., the cumulative workload during interval $j$, denoted $x^j$, may be distributed amongst the network resources such that each network resource gets a fraction of the workload dentoed $x_i^j$. In such networks, the sum of $x_i^j$ over all network resources must equal $x^j$. As discussed in section~\ref{sec:background:differences}, workload mapping in geo-diverse data centers may be approximated using such a scheme\footnote{The number of client requests per interval is so large that the fractional distribution (to a reasonable precision) of workload amongst resources results in a solution that will most likely have an integer number of requests mapped to each data center}. One would expect that the resulting problem would be simpler to solve and shouldn't be NP-Hard, because the single interval instance now resembles the fractional knapsack problem which may be solved optimally using a greedy strategy. But as we shall see, when RP is applied in a multi-interval setting, even this version of the problem is NP-Complete.

We compare this second version of our problem to the unit commitment problem~\cite{unitcommit-trans} in distributed electricity generation and distribution scenario, which is known to be NP-Complete. The unit commit problem determines the generation levels of several power generating resources, given time-varying demand for electricity that may be derived from any of the active generating resources in any fraction. The generating resources may be turned off when electricity demand is lower than the cumulative capacity of running generating resources while incurring a ramp-down cost. Similarly, a generating resource may be turned on when demand exceeds the capacity of resources that are currently operating while incurring a ramp-up cost. Ramp-up and ramp-down costs as well as costs of generating a unit of electricity at each generating resource depends on the fuel prices at the location where the corresponding generating resource is situated. Furthermore, a generator is allowed to run on no-load as a spinning reserve while incurring idle fuel costs. If we represent the time-varying electricity demand as the network operator's workload and replace the generating resources by data centers, we have a one-to-one mapping of the unit commit problem to our geo-diverse data center scenario. Since the unit commitment problem is NP-Complete, it follows that so is the fractional workload-mapping version of our problem.

\section{Optimization problem formulation}
\label{sec:framework:optimization}
On a high-level, routine network operation (in the context of our thesis) involves distributing workload to network resources and periodically updating the fraction of workload mapped to each network resource. For simplicity of modeling and analysis, we assume that the mapping of workload to network resources, henceforth referred to as \textit{workload mapping} or simply \textit{mapping}, is updated at the beginning of intervals of fixed duration. We define resource $i$'s state during interval $j$, denoted $s_i^j$, as the corresponding resource's status (on or off) and the amount of workload mapped to it. The aggregated state of all network resources during interval $j$ may be termed as the \textit{network state} during the corresponding interval, denoted by $S^j$. The routine network operation can thus be modeled as determing a sequence of states for a time horizon, called a \textit{planning window}, consisting of a set of consecutive intervals of equal duration. In the context of our thesis, the objective is to determine the state trajectory that is optimal in the sense that it minimizes the electricity cost over the planning window. %The interval after which the workload mapping may change need not be less than the interval during which the workload and electricity prices remain almost constant. Both problems, i.e., mapping real fractions of workload to network resources (as in the geo-diverse data center scenario) and the 0/1 mapping of workload (as in the cellular network scenario) are multi-interval allocation 

\subsection{The objective function}
\label{subsec:framework:objective} %Provide the mathematical form of the objective function that is designed to solve the optimal state trajectory problem.
The optimal state trajectory problem attempts to determine a sequence of states such that the sum of state costs and the cost of transitions between states in consecutive intervals is minimized. Mathematically, we may present the objective function as:

\begin{align}
\sum_{j=1}^n C(S^j) + T(S^j, S^{j-1})
\end{align}

Here $C(S^j)$ represents a function that evaluates the cost of being in state $S^j$ and $T(S^j,S^{j-1})$ represents the cost of transitioning from state $S^{j-1}$ to state $S^j$. In the context of our thesis, the function $C(S^j)$ should evaluate the electricity cost of network operation given that the network is in state $S^j$. Similarly, $T(S^j,S^{j-1})$ should compute the cost of changing network state between two consecutive intervals that may arise due to factors such as turning network resources on or off.

\subsection{The constraints}
\label{subsec:framework:constraints} %Comment on some of the network-specific constraints that the optimization must be subject to.
The state trajectory problem must be subject to a number of problem-specific constraints. Some constraints are common to all types of networks. 

\begin{itemize}
\item \textbf{Resource capacity must be respected:} During all intervals, we must ensure that the workload mapped to each network resource does not exceed it's capacity.
\item \textbf{All workload must be handled:} During all intervals, the sum of workload mapped to all network resources must equal the offered workload for that interval.
\item \textbf{Network resource status is binary:} The status of a network must be represented as a binary variable which takes on the value 1 if the said network resource is on during a given interval, and 0 otherwise. We can not simply represent the resource state using a real-valued variable, representing the fraction of the current cumulative workload mapped to the corresponding resource, because it is not possible to determine instances of activation/deactivation of resources using such real-valued variables.
\end{itemize}

Several network-specific constraints must also be formulated. These constraints arise from subtle differences between different types of network. For instance, while any client request can be handled at any data center, a given call may be handled by only a few BTSs that are in the immediate vicinity of a caller.

\subsection{Comments on the problem formulation}
The decision variables in our problem formulation are the state of network resources for each interval in the planning window. A resource's state has two parts: the amount of workload it handles and it's status (on or off). The resource status needs to be a discrete (binary) variable. The amount of workload may also be a whole number in some network types. For instance, in a cellular network context, the workload mapped to a resource represents the number of active calls being handled by a BTS. In some other cases, on the other hand, such as the geo-diverse data center scenario, the amount of workload mapped to a resource may be represented using a real-valued variable, representing the fraction of total workload being handled by a data center during a given interval. In the first scenario, such as cellular networks, the resource state is purely discrete, whereas in the second scenario, such as geo-diverse data centers, the resource state is composed of a discrete as well as real-valued parts. In the former case, our optimization problem is an integer program (IP), whereas in the latter, the problem is a mixed integer program (MIP). Both IP and MIP are NP-Hard, however, and must be solved using techniques such as branch and bound~\cite{land60a} or other heuristics. If functions $C(.)$ and $T(.,.)$ as well as all constraints are linear and convex, the formulation is termed as an integer linear program or mixed integer linear program. Since the branch and bound technique repeatedly solves constrained and integer-relaxed versions of the IP (or MIP), having linear objective functions results in lower computational complexity. Fortunately, the nature of energy consumption in networks is such that the power consumption function $C(.)$ is linear and convex. For this reason, in our thesis, we strive to make the transition cost function $T(.,.)$ as well as all problem constraints as linear.

In this chapter, we have presented only an abstract formulation of the general optimization problem for minimizing electricty cost in service provider networks. We find it appropriate to comment here on the more concrete mathematical formulation of the problem in specific network types as well as the techniques used to solve those concrete instances of the problem. Two concrete instances of the optimization problem are discussed in the next two chapters. In order to solve both of those concrete instances, we used the CPLEX solver, which uses the branch and bound heuristic, available with the ILOG CPLEX Studio which is available free of cost for academic use. Our primary focus in this thesis is to investigate the potential that the use of RP and WR offers for electricity cost optimization. Therefore, we focus on solving both concrete instances of the optimization problem \textit{exactly} rather than proposing heuristics for approximate solutions to the problem\footnote{We do, however, propose a heuristic for the optimization problem in a cellular network setting.}. As we shall see in the next two chapters, we were able to solve problems of reasonable size using simple desktop PCs within a reasonable amount of time.