\chapter{A generalized framework for electricity cost optimization}
\label{chap:framework} In Chapter~\ref{chap:intro}, we have seen that networks, such as geo-diverse data centers and cellular networks, are energy inefficient. This is due to lack of energy proportionality coupled with significant variations in workload. We have also observed that these networks are plagued by high electricity costs. For a given electricity price, if these networks were energy efficient, high electricity costs would only be due to heavy workload. However, energy inefficient networks consume a lot more energy than they ideally should. Hence, high network electricity costs are a source of concern. 

In Chapter~\ref{chap:background}, we developed insights into how these networks operate in order to get a better understanding about their power consumption model. We also saw the similarities and differences between different network types, related to power consumption, that must be kept in mind. in this chapter, we will use this knowledge to formulate an optimization problem that minimizes electricity costs for different types of networks.


\section{Modeling the electricity cost minimization problem in networks and systems} %Discuss how different network types can periodically use RP and WR to minimize electricity costs. Show that this problem is NP-Hard. Describe the concept of network state and motivate a state trajectory optimization problem. Describe transition costs and formulate a mathematical optimization problem. 

Let us illustrate network operation from the standpoint of electricity consumption and cost using an example shown in Figure~\ref{fig:mappingexample}. The example uses a test tube to represent a network resource and marbles to represent a unit workload. The network resource could be a data center in the context of geo-diverse data center operator, whereas it could be a transceiver in the case of a cellular operator. Similarly, the workload unit could be a client request in the data center context, whereas it could be a call in a cellular network setting. The operator's goal is to assign workload to network resources and, if needed, periodically update this mapping in response to variations in workload.

We consider the largest possible quantum of time for which the workload (and electricity price) remains fixed and term each quantum as an \textit{interval}. We assume that workload for several consecutive intervals is known and term this sequence of intervals as a planning window.   The example demonstrates three different ways of mapping this workload to two network resources situated at different locations. For simplicity we assume in this example that the workload is geographically split such that half of it originates near each of the two resource. Figure~\ref{fig:mappingexample} (a) shows the operator workload over a planning window consisting of three consecutive intervals. Meanwhile, Figure~\ref{fig:mappingexample} (b) shows the geo-temporal variation in electricity prices for the two network resources.


One possible operational strategy is to map each workload unit to the nearest available resource as shown in Figure~\ref{fig:mappingexample} (c). In a sense, this is the default strategy in cellular networks, whereby a call is handled by the BTS from which the mobile station (MS) receives the strongest radio signal\footnote{Signal from the physically nearest BTS may be weakened considerably due to natural or man-made obstructions. In such cases, the nearest BTS may not be the one from which the strongest signal is received. Hence, we take "nearest" to mean the BTS from which the MS receives the strongest signal}. In geo-diverse data center settings, this sort of mapping is also often the default strategy because it minimizes the access latency for all clients\footnote{Network latency has been shown to have a strong correlation with the physical shortest path distance between two locations on the globe~\cite{dina:p2pdelay:infocom:2004}. So, the commonly understood physical measure of "shortest" applies in this case.}. 

The above workload-resource mapping strategy pays no attention to electricity prices. We can reduce the electricity cost over the planning window by mapping more workload to resources at cheaper locations. We term the change in workload-resource mapping as Workload Relocation (WR). Figure~\ref{fig:mappingexample} (d) shows a mapping strategy which uses WR to map as much workload as possible to resources at cheaper locations. In interval $t_1$, since the cumulative workload equals the total network capacity, both network resources will be operating at capacity. In interval $t_2$, on the other hand, we may use WR to move all workload to the network resource at the location with the cheapest electricity price (subject to resource capacity constraints, of course), thereby reducing electricity cost for that interval. In interval $t_3$, again, the workload may be shifted to the network resource at the location with cheapest electricity price during that interval.

Due to lack of energy proportionality in networks, the power consumption of idle resources is a large fraction of their peak power consumption. Hence, consolidation of workload to cheaper locations has limited benefit in terms of electricity cost reduction. To have considerable savings in electricity cost, one must use RP, i.e., deactivate idle resources. Notice that in the deafult workload-resource mapping strategy of Figure~\ref{fig:mappingexample} (c), there is no opportunity to deactivate idle resources. However, if WR is combined with RP, as shown in Figure~\ref{fig:mappingexample} (e), maximal savings in electricity cost can be achieved, because it not only shifts workload to the cheapest possible resources, but also deactivates as many resources as possible. 

\begin{figure}
\centering
\includegraphics[height=0.7\textheight]{pics/mappingexample.eps}
\caption{An example of mapping variable workload to capacity-limited network resources with geo-temporal diversity in electricity prices. Three consecutive intervals $t_1$, $t_2$ and $t_3$ are considered. Workload and electricity prices may only change between two consecutive intervals. (a) Workload considered in this example. (b) Electricity prices for the locations at which the two network resources are situated. (c) A uniform mapping of workload to network resources does not exploit electricity price diversity. (d) Mapping workload to network resources in order of their current electricity price. Due to lack of energy proportionality, only slight savings in electricity cost are possible. (e) Deactivating idle resources alongwith the resource mapping strategy of (d) may result in significant electricity cost savings.}
\label{fig:mappingexample}
\end{figure}

During each interval, the problem is a generalization of the multiple knapsacks problem ~\cite{kellerer:knapsackproblems:2005,Chekuri99aptas}, which is known to be NP-Hard. Since the single-interval instance of our problem, in this case, is NP-Hard, so is the multi-interval version. For a single instance, the problem becomes tractable if a unit workload could be divided into fractions amongst the network resources. For instance, in the case of geo-diverse data centers, the number of client requests per interval is so large that the fractional distribution (to a reasonable precision) of workload amongst resources results in a solution that will have an integer number of requests mapped to each data center. But as we shall see, when RP is applied in a multi-interval setting, the problem is still NP-Hard.

In a multi-interval setting, we compare our problem to the unit commitment problem~\cite{unitcommit} in smart grids and show that it is NP-Complete. The unit commit problem determines the generation levels of several power generating resources, given time-varying demand for electricity that may be derived from any of the active generating resources in any fraction. The generating resources may be turned off when electricity demand is low and turned on when demand exceeds the capacity of resources that are currently operating. The turning on and off incur costs as the generating resource must be ramped-up and ramped-down. If we 


\subsection{The objective function} Provide the mathematical form of the objective function that is designed to solve the optimal state trajectory problem.
\subsection{The constraints} Comment on some of the network-specific constraints that the optimization must be subject to.

