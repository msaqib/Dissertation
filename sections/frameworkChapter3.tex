\chapter{A generalized framework for electricity cost optimization}
\label{chap:framework} In the last chapter, we have seen that networks, such as geo-diverse data centers and cellular networks, are energy inefficient. This is due to lack of energy proportionality coupled with significant variations in workload. We have also observed that these networks are plagued by high electricity costs. For a given electricity price, if these networks were energy efficient, high electricity costs would only be due to heavy workload. However, energy inefficient networks consume a lot more energy than they ideally should. Hence, high network electricity costs are a source of concern. 

Let us illustrate the problem from an operator's perspective using an example shown in Figure~\ref{fig:mappingexample}. The example uses a test tube to represent a network resource and marbles to represent a unit workload. The network resource could be a data center in the context of geo-diverse data center operator, whereas it could be a transceiver in the case of a cellular operator. The workload unit could be a client request in the data center context, whereas it could be a call in a cellular network setting. The operator's goal is to assign workload to network resources and this mapping might need to be changed as workload volume changes.

We consider the largest possible quantum of time for which the workload (and electricity price) remains fixed and term each quantum as an \textit{interval}. We assume that workload for several consecutive intervals is known and term this sequence of intervals as a planning window. Figure~\ref{fig:mappingexample} (a) shows the operator workload over a planning window consisting of three consecutive intervals. For simplicity we assume for the purpose of this example that the workload is geographically split such that half of it originates near each of the two resource. The example demonstrates three different ways of mapping this workload to two network resources situated at two different locations. Figure~\ref{fig:mappingexample} (b) shows the geo-temporal variation in electricity prices for the two network resources.


A simple way to map the workload to network resources is to map each workload unit to the nearest available resource. Figure~\ref{fig:mappingexample} (c) shows such a strategy. This strategy is the default in cellular networks, whereby a call is handled by the BTS from which the mobile station (MS) receives the strongest radio signal. In geo-diverse data center settings, this sort of mapping corresponds to minimizing the access latency for all clients. This strategy pays no attention to electricity prices. We can reduce the electricity cost over the planning window by mapping more workload to resources at cheaper locations. We term the change in workload-resource mapping as Workload Relocation (WR). Figure~\ref{fig:mappingexample} (d) shows a mapping strategy which uses WR to map as much workload as possible to resources at cheaper locations. In interval $t_1$, since the cumulative workload equals the total network capacity, there is nothing we can do, except fill all resources to 

One way to deal with this energy inefficiency is what we call resource pruning (RP). The idea is to deactivate some network resources when they are not needed. Since the workload is variable, the operator must periodically determine the optimal network configuration, i.e., the active/inactive state of each network resource and the mapping of each unit of workload to 

\begin{figure}
\centering
\includegraphics[height=0.7\textheight]{pics/mappingexample.eps}
\caption{An example of mapping variable workload to capacity-limited network resources with geo-temporal diversity in electricity prices. Three consecutive intervals $t_1$, $t_2$ and $t_3$ are considered. Workload and electricity prices may only change between two consecutive intervals. (a) Workload considered in this example. (b) Electricity prices for the locations at which the two network resources are situated. (c) A uniform mapping of workload to network resources does not exploit electricity price diversity. (d) Mapping workload to network resources in order of their current electricity price. Due to lack of energy proportionality, only slight savings in electricity cost are possible. (e) Deactivating idle resources alongwith the resource mapping strategy of (d) may result in significant electricity cost savings.}
\label{fig:mappingexample}
\end{figure}

\section{Modeling the electricity cost minimization problem in networks and systems} Discuss how different network types can periodically use RP and WR to minimize electricity costs. Show that this problem is NP-Hard. Describe the concept of network state and motivate a state trajectory optimization problem. Describe transition costs and formulate a mathematical optimization problem. 
\subsection{The objective function} Provide the mathematical form of the objective function that is designed to solve the optimal state trajectory problem.
\subsection{The constraints} Comment on some of the network-specific constraints that the optimization must be subject to.

