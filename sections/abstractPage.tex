\begin{abstract}

It is hard to imagine living for a single day, in the present age, without networks. Our reliance, for day to day operations, is on telephone networks, cellular networks and the Internet. %and large scale geo-diverse data centers%Not an end-user's perspective. 
These networks consumes a lot of electric energy annually. This results in tens of millions of dollars worth of operational expenditure on electricity costs. 

Existing approaches to energy efficiency improvement may be categorized into incremental vs clean-slate approaches. In the clean-slate approach, radical new designs and architectures are proposed that are energy efficient but incompatible with existing networks. In the incremental approach, an attempt is made to improve energy efficiency while remaining within the confines of the existing standards and hardware. In this thesis, we follow the incremental approach.

% Read abstract/intro from all three papers and find the least common denominator. Forward seems OK

In this thesis, we identify abstractions in the power consumption models of these two types of networks. Based on these abstractions, we were able to formulate a general power consumption model for these networks. Using this general model, we identify two operations, namely, resource pruning and workload relocation that can be used to improve the energy efficiency and reduce the electricity consumption. We formulate the energy efficiency improvement problem as a multi-interval optimal state trajectory problem, called RED-BL: Relocate Energy Demand to Better Locations.

We evaluated the benefit of RED-BL using real datasets obtained from geo-diverse data centers as well as cellular networks. Our results indicate that significant savings in electricity consumption and cost may be obtained by the application of RED-BL to these types of networks. In case of geo-diverse data centers, RED-BL can reduce electricity costs by as much as 45\%. In case of cellular networks, the energy savings were as high as 22\%.


\end{abstract}

%\begin{keywords}

%Charles Sanders Peirce, evolutionary algorithms, non Darwinian
%theories of evolution, stagnation, epigenetics, systems biology.

%\end{keywords}
