\begin{abstract}

%It is hard to imagine a single day in our lives when we do not use services that rely on networks. We make telephone calls, interact with friends, family members and colleagues using email and online social networks. We access these services by connecting to telephone networks, cellular networks and Internet service provider networks. There are other networks that run in the background. For instance, many Internet-based services such as hosted email, online social networks and online storage services are enabled by networks of geo-diverse data centers. All of these networks consume a huge amount of electric energy. For instance, cellular networks worldwide consume several tens of Terra Watt Hours (TWhs) of energy annually. A single data center's power consumption may be of the order of 10 MW. The US Environmental Protection Agency (EPA) reports that data centers consumed 2-3\% of total electrical energy in the USA. Consequently, a network service provider spends tens of millions of dollars annually on electricity costs. Thus, service providers are keen on reducing the electricity consumption in their networks.

%For several service providers, such as cellular network operators and data center operators, the workload is time-varying and has diurnal cycles. Furthermore, the workload peaks for only a short duration each day before falling off to a much lower trough. To cater to customer demand, however, the service provider must dimension the network according to peak workload. This means deploying as many resources\footnote{A server is an example of network resource in case of network of geo-diverse data centers, whereas a transceiver is an example of resource in case of cellular networks.} in the network as are needed to handle peak workload. These network resources lack energy proportionality, i.e., the power consumption does not scale in proportion to the workload. Consequently, the networks also lack energy proportionality and consume electricity at about the same level as the peak power consumption. This leads to wasted electric energy, which this thesis aims to reduce.

%Fine-grained load-proportionality can be achieved by using network resources that are energy-proportional, which is not possible using the state of the art. However, coarse-grained proportionality can still be achieved by keeping as many network components off or in power-saving state as possible without compromising handling of current workload. This results in lowered power consumption during low-workload regimes. We term this strategy as resource pruning. Furthermore, by smartly distributing workload among network components, the number of network components turned off or in power-saving state may be maximized. We term this strategy as workload relocation.

Service provider networks enable services that we rely on for many essential everyday functions. These networks are composed of several network resources\footnote{such as servers, radio transceivers, network links and routers}, each of which has a maximum customer workload handling capacity. The operators dimension the networks with enough network resources so that they may handle the expected peak of the cumulative customer workload. The customer workload is time-varying and has a large peak-trough ratio. Since the network resources lack energy proportionality, these networks always consume electricity at about the same level as the peak power consumption. This leads to wasted electric energy, which this thesis aims to reduce.

We propose saving electricity by using a two-pronged strategy. First, we reduce power consumption during low workload regimes by keeping as many network components off or in power-saving state as possible without compromising handling of current workload. Secondly, by smartly distributing workload among network components, the number of network components turned off or in power-saving state may be maximized. We term these strategies as resource pruning and workload relocation, respectively.

Both resource pruning and workload relocation control the state of network resources, which in turn determines the network's power consumption. We may consider the aggregate of the instantaneous state(on, off, power-saving and current workload assigned to each resource) of all network resources as the network's state. Due to workload variations, no single network state can be optimal for a network all the time. Therefore, we formulate the energy efficiency improvement problem as a multi-interval optimal state trajectory problem, called RED-BL: Relocate Energy Demand to Better Locations. RED-BL computes the optimal states for a network for a time horizon based on workload estimates, future electricity prices and the cost of transition between network states during consecutive intervals.

We evaluated the benefit of RED-BL using real datasets obtained from geo-diverse data centers as well as cellular networks. Our results indicate that significant savings in electricity consumption and cost may be obtained by the application of RED-BL to these types of networks. In case of geo-diverse data centers, RED-BL can reduce electricity costs by as much as 45\%. In case of cellular networks, the energy savings were as high as 22\%.


\end{abstract}

%\begin{keywords}

%Charles Sanders Peirce, evolutionary algorithms, non Darwinian
%theories of evolution, stagnation, epigenetics, systems biology.

%\end{keywords}
