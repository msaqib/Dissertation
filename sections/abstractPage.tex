\begin{abstract}

Service provider networks enable services that we rely on for many essential everyday tasks. These networks are shared by many users and must be able to handle the cumulative workload from all the users at any given time. These networks are composed of several network resources\footnote{such as servers, radio transceivers, network links and routers}, each of which has a maximum workload handling capacity. Operators, therefore, dimension networks with enough network resources so that they may handle the expected peak of the cumulative customer workload. The customer workload is time-varying and has a large peak-trough ratio. Since network resources lack energy proportionality, networks always consume electricity at about the same level as the peak power consumption. This leads to wasted electric energy, which this thesis aims to reduce.

We propose saving electricity by using a two-pronged strategy. First, we reduce power consumption during low workload regimes by keeping as many network components off, or in power-saving state, as possible without compromising handling of current workload. Secondly, by smartly distributing workload among network components, we aim to maximize the number of network components turned off or in power-saving state. We term these strategies as resource pruning and workload relocation, respectively.

Both resource pruning and workload relocation control the state of network resources, i.e., on/off/power-saving and current workload assigned to each resource. The network resource state, in turn, determines the network's power consumption. We may consider the aggregate of the instantaneous state  of all network resources as the network's state. Due to workload variations, no single network state can be optimal for a network at all times. Therefore, we formulate the energy efficiency improvement problem as a multi-interval state trajectory optimization, called RED-BL: Relocate Energy Demand to Better Locations. RED-BL computes the optimal states for a network over a time horizon by using workload estimates, future electricity prices and the cost of transition between network states during consecutive intervals.

We evaluated the benefit of RED-BL using real datasets obtained from geo-diverse data centers as well as cellular networks. Our results indicate that significant savings in electricity consumption and cost may be obtained by the application of RED-BL to these types of networks. In case of geo-diverse data centers, RED-BL can reduce electricity costs by as much as 45\%. In case of cellular networks, the energy savings were as high as 22\%.


\end{abstract}

%\begin{keywords}

%Charles Sanders Peirce, evolutionary algorithms, non Darwinian
%theories of evolution, stagnation, epigenetics, systems biology.

%\end{keywords}
